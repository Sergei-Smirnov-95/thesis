%%%%%%%%%%%%%%%%%%%%%%%%%%%%%%%%%%%%%%%%%%%%%%%%%%%%%%%%%%%%%%%%%%%%%%%%%%%%%%%%
\intro
%%%%%%%%%%%%%%%%%%%%%%%%%%%%%%%%%%%%%%%%%%%%%%%%%%%%%%%%%%%%%%%%%%%%%%%%%%%%%%%%

В настоящее время все большее распространение получают различные приложения и системы. Социальные сети, интернет-магазины и всевозможные онлайн сервисы окружают современного человека повсюду. Каждое такое приложение имеет тысячи, миллионы,а иногда и миллиарды пользователей. И в каждой такой системе хранятся всевозможные данные: фотографии,  аудио и видеофайлы, текстовая информация. При ежедневном использовании таких систем, мало кто задумывается о том, что все эти данные, в количестве, зачастую превышающем петабайты, нужно где-то хранить. Для этих целей мировые бренды разрабатывают и модифицируют системы хранения данных. Наиболее важными критериями оценки~\cite{reliability} вычислительных систем в целом и систем хранения данных(СХД) в частности являются: 
\begin{itemize*}
	\item{Надежность}
	\item{Отказоустойчивость}
	\item{Скорость доступа к данным}
\end{itemize*}	
Значения критериев скорости доступа к данным и надежности закладываются на этапе создания и доработки систем. Повышение отказоустойчивости системы может достигаться следующими путями:
\begin{itemize*}
	\item{Применение избыточных компонентов охлаждения, блоков питания}
	\item{Использование алгоритмов коррекции ошибок памяти}
	\item{Использование средств мониторинга состояния системы}
	\item{Введение функций проверки состояния системы и ее компонентов}
\end{itemize*}	

В рамках проекта при поддержке Министерства науки и высшего образования Российской Федерации  в рамках Федеральной целевой программы «Исследования и разработки по приоритетным направлениям развития научно-технологического комплекса России на 2014-2020 годы»(уникальный идентификатор проекта  – RFMEFI58117X0023) лабораторией разрабатывается система мониторинга и диагностики состояния СХД.

В рамках данной работы будет рассмотрен ряд прикладных задач, имеющих непосредственное применение в разарбатываемой системе. 

Все собираемые параметры при мониторинге какой-либо системы можно объединить в две группы: 
\begin{itemize*}
	\item{Внутренние параметры;}
	\item{Внешние параметры.}
\end{itemize*}	

Внутренние параметры -это параметры, получаемые системой мониторинга от сервисов, расположенных в наблюдаемой системе. Это могут быть журналы событий, логи, статистика с запускаемых утилит и пр. К числу таких параметров относятся SMART-параметры диска. В данной работе описан анализ наиболее полезных для диагностики состояния диска параметров а также показана методология диагностирования состояния дисковой подсистемы СХД.

Внешними параметрами являются климатические, сейсмические и прочие данные, получаемые извне системы, но влияющие на состояние системы. В рамках данной работы интерес представляют климатические параметры, как параметры, оказывающие влияние на состояние серверной стойки и дисков в ней. 

Несмотря на большое количество реализаций систем сбора климатических параметров, универсального решения, подходящего под любую задачу, не существует. Каждая реализация эффективна только лишь в своей нише, разрабатываемая под свои цели. Перспективной задачей представляется разработка универсальной системы сбора и отображения данных о климатических параметрах с возможностью ее использования как части мультисенсорной системы сбора параметров.  В рамках данной работы формулируется задача проектирования и построения прототипа программно-аппаратного комплекса сбора и отображения климатических параметров в части реализации для применения в СХД. 
Для достижения поставленной цели необходимо решить следующие задачи:
\begin{itemize*}
	\item{Разработка структурной схемы комплекса;}
	\item{Разработка и реализация аппаратной части;}
	\item{Разработка и реализация программной части.}
\end{itemize*}	
По результатам работы совместно с Успенским М.Б. была подана научная статья "Обзор подходов к обнаружению сбоев в системах хранения данных" в журнал "Научно-технические ведомости СПбПУ. Информатика. Телекоммуникации. Управление". Разработанный АПК используется для сбора и анализа статистических данных о работе СХД в рамках ФЦП № RFMEFI58117X0023. 
Кроме того, совместно с Успенским М.Б. был зарегистрирован РИД "Программа для сбора и отображения климатических параметров систем хранения данных", свидетельство № 2019614476. 
