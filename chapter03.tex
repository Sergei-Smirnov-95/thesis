%%%%%%%%%%%%%%%%%%%%%%%%%%%%%%%%%%%%%%%%%%%%%%%%%%%%%%%%%%%%%%%%%%%%%%%
\chapter{Разарботка структуры АПК сбора климатических параметров}
%%%%%%%%%%%%%%%%%%%%%%%%%%%%%%%%%%%%%%%%%%%%%%%%%%%%%%%%%%%%%%%%%%%%%%%
В рамках командного курсового проекта во втором семестре магистратуры был разработан прототип АПК сбора климатических параметров(рис. ~\ref{fig:proto}).

\begin{figure}[H]
	\centering
	\includegraphics[width=\textwidth]{proto}
	\caption{Прототип АПК сбора климатических параметров СХД}
	\label{fig:proto}
\end{figure}
Данный прототип представляет из себя плату управления и набор датчиков, непосредственно подключенных к ней. В ходе исследовательских испытаний был выявлен ряд недостатков: 
\begin{itemize*}
	\item{Отсутствие общей платы сопряжения}
	\item{Отсутствие корпуса изделия}
	\item{Отсутствие единого блока питания}
	\item{Отсутсвие дисплея}
\end{itemize*}

Исполнение прототипа навесным монтажом обернулось постоянной проблемой отпадающих контактов, что затрудняло испытания. Кроме того, в схеме не предусмотрены подтягивающие резисторы для цифровых каналов связи, что создавало дополнительные помехи. 

Отсутствие встроенного блока питания и корпуса изделия не сказывалось на использование прототипа, однако для реального использования АПК как единицы сбора данных необходимо снабдить прототип и блоком питания и корпусом. Стоит отметить, что изменение формы и вида корпуса позволит гибко использовать прототип в различных целях.

Также, в ходе испытаний прототипа было выявлено, что при использовании прототипа в качестве системы сбора климатических параметров в стойке СХД, актуальным является отображение текущих измеряемых АПК показаний на дисплее. 

С точки зрения программного обеспечения, разработанный протоип представляет из себя набор драйверов, позволяющих обращаться к конкретному датчику и получать с него показания. Для использования прототипа как части системы сбора данных, требуется реализовать REST сервер, осуществляющий ответы на запросы текущих показаний датчиков, а также исторических показаний за период равный посленим суткам. Кроме того, для удобства отслеживания изменения параметров требуется разработать web интерфейс, отображающий текущие и исторические параметры в виде графиков.

Дополнительная аппаратная задача по добавлению дисплея для отображения текущих показаний добавляет программную задачу работы с дисплеем. 
%%%%%%%%%%%%%%%%%%%%%%%%%%%%%%%%%%%%%%%%%%%%%%%%%%%%%%%%%%%%%%%%%%%%%%%
\section{Разработка аппаратного комплекса}
%%%%%%%%%%%%%%%%%%%%%%%%%%%%%%%%%%%%%%%%%%%%%%%%%%%%%%%%%%%%%%%%%%%%%%%

%%%%%%%%%%%%%%%%%%%%%%%%%%%%%%%%%%%%%%%%%%%%%%%%%%%%%%%%%%%%%%%%%%%%%%%
\section{Разработка программного комплекса}
%%%%%%%%%%%%%%%%%%%%%%%%%%%%%%%%%%%%%%%%%%%%%%%%%%%%%%%%%%%%%%%%%%%%%%%








%%You can use all kinds of abbreviations that don't mean anything, but add
%%a false sense of importance and significance to your work. Some of these
%%abbreviations are:
%
%%\begin{itemize*}
%%\item eXtensible Markup Language~(XML)
%%\item JavaScript Object Notation~(JSON)
%%\item Yet Another Markup Language~(YAML)
%%\end{itemize*}
%


%%\begin{table}
%%\captionsetup{skip=5pt}
%%\caption{Решетка замечательности аббревиатур}
%%\centering
%%XML < JSON < YAML
%%\end{table}


%%%%%%%%%%%%%%%%%%%%%%%%%%%%%%%%%%%%%%%%%%%%%%%%%%%%%%%%%%%%%%%%%%%%%%%%%%%%%%%%
%%\section{}
%%%%%%%%%%%%%%%%%%%%%%%%%%%%%%%%%%%%%%%%%%%%%%%%%%%%%%%%%%%%%%%%%%%%%%%%%%%%%%%%

