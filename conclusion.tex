%%%%%%%%%%%%%%%%%%%%%%%%%%%%%%%%%%%%%%%%%%%%%%%%%%%%%%%%%%%%%%%%%%%%%%%%%%%%%%%%
\conclusion
%%%%%%%%%%%%%%%%%%%%%%%%%%%%%%%%%%%%%%%%%%%%%%%%%%%%%%%%%%%%%%%%%%%%%%%%%%%%%%%%
Одним из наиболее эффективных способов повышения отказоустойчивости систем хранения данных является диагнсотика таких систем, и, как следствие, мониторинг параметров в таких системах. 
В данной работе описана методика диагностирования состояний СХД на примере жестких дисков с использованием климатических параметров.

Кроме того, результатом работы является изготовленый прототип аппаратно-программного комплекса: аппаратной части(фотографии изделия представлены на рис. ~\ref{fig:result}) и программной части(внешний вид веб-интерфейса представлен на рис. ~\ref{fig:web} -~\ref{fig:web2}).

По результатам работы совместно с Успенским М.Б. была подана научная статья "Обзор подходов к обнаружению сбоев в системах хранения данных" в журнал "Научно-технические ведомости СПбПУ. Информатика. Телекоммуникации. Управление". Разработанный АПК используется для сбора и анализа статистических данных о работе СХД в рамках ФЦП № RFMEFI58117X0023. 
Кроме того, совместно с Успенским М.Б. был зарегистрирован РИД "Программа для сбора и отображения климатических параметров систем хранения данных", свидетельство № 2019614476. 

Дальнейшее развитие проекта возможно в направлении расширения сбора статистики влияния параметров на SMART параметры дисков.