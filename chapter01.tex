%%%%%%%%%%%%%%%%%%%%%%%%%%%%%%%%%%%%%%%%%%%%%%%%%%%%%%%%%%%%%%%%%%%%%%%%%%%%%%%%
\chapter{Анализ систем мониторинга и постановка задачи}
%%%%%%%%%%%%%%%%%%%%%%%%%%%%%%%%%%%%%%%%%%%%%%%%%%%%%%%%%%%%%%%%%%%%%%%%%%%%%%%%
В настоящее время существует большое количество программных комплексов
мониторинга систем в различных областях. В рамках данной работы интерес представляют системы, использование которых было бы целесообразно для проведения мониторинга СХД.
Средства, с помощью которых можно осуществлять мониторинг параметров систем хранения данных можно разделить на две группы:
\begin{itemize*}
	\item{Специальные системы}
	\item{Системы мониторинга общего назначения}
\end{itemize*}
Рассмотрим подробнее каждую группу.
 
%%%%%%%%%%%%%%%%%%%%%%%%%%%%%%%%%%%%%%%%%%%%%%%%%%%%%%%%%%%%%%%%%%%%%%%
\section{Специальные системы мониторинга}
Представителями первой группы являются продукты фирм IBM, EMC и FUJITSU. Данные системы разработаны самими  производителями систем хранения данных и чаще всего поставляются в комплекте с СХД.

\subsection{IBM Spectrum}
Система мониторинга параметров СХД поставляемая с оборудованием IBM. Позволяет нотифицировать об изменении состояния компонента и отслеживать состояние компонентов визуально. Пользовательский интерфейс системы представлен на рисунке~\ref{fig:Spectrum}.
\begin{figure}[!h]
	\centering
	\includegraphics[width=\textwidth]{Spectrum}
	\caption{Пользовательский интрефейс IBM Spectrum}
	\label{fig:Spectrum}
\end{figure}

В системе предполагается градация по пяти уровням критичности произошедшего события~\cite{Spectrum}. Уровни градации представлены на рисунке~\ref{fig:Spectrum2}.
\begin{figure}[!h]
	\centering
	\includegraphics[width=\textwidth]{Spectrum2}
	\caption{Градация сообщений системы Spectrum}
	\label{fig:Spectrum2}
\end{figure}
Среди достоинств системы можно отметить широкую классификацию событий и понятный пользователю интерфейс. Однако, система не обладает возможностью присылать уведомления  админимтратору на почту/смс, а значит требует постоянного присутствия человека. Вторым недостатком системы является большая ориентированность на скоростные, стоимостные и нагрузочные показатели, нежели на характеристику компонентов и системы в целом. 
\subsection{EMC VNX Monitoring and Reporting}
Система мониторинга и информирования от EMC~\cite{EMC}, предоставляемая совместно с оборудованием. Позволяет в графическом виде наблюдать структуру системы, нагрузку на компоненты, активность сервисов системы. 
\begin{figure}[!h]
	\centering
	\includegraphics[width=\textwidth]{EMC}
	\caption{Пользовательский интрефейс EMC VNX}
	\label{fig:EMC}
\end{figure}
Аналогично системе Spectrum, используются информирующие сообщения пяти уровней критичности, но имеется возможность уведомления о событиях на электронную почту. Кроме того, в системе присутствует большое количество сетевых и системных настроек, позволяющих создавать кластеры, размечать диски, настраивать доступы и сети. Таким образом, система является в большей степени комплексной,а задача мониторинга состояния компонентов не является основопологающей. 

\subsection{FUJITSU ServerView System Monitor}
Еще одна специальная система мониторинга разработана компанией FUJITSU~\cite{Fujitsu}. Является визуализатором дерева подкомпонентов системы с состоянием каждого. Пользовательский интерфейс представлен на рисунке~\ref{fig:Fujitsu}.  
\begin{figure}[!h]
	\centering
	\includegraphics[width=\textwidth]{Fujitsu}
	\caption{Пользовательский интрефейс FUJITSU Server View}
	\label{fig:Fujitsu}
\end{figure}
Кроме десктопного исполнения FUJITSU разработала мобильное приложение, с аналогичной функциональностью (см. рисунок ~\ref{fig:Fujitsu2}).
\begin{figure}[!h]
	\centering
	\includegraphics[width=2.0in]{Fujitsu2}
	\caption{Мобильное приложение от Fujitsu}
	\label{fig:Fujitsu2}
\end{figure}

Преимуществом данной системы по сравнению с ранее рассмотренными является большее уделение внимания состоянию системы в целом и компонентов в частности. Дерево компонентов системы позволяет быстро оценить состояние каждого,а при необходимости раскрыть и узнать большее количество информации, вплоть до причины возникновения ошибки. Кроме того, разработанное мобильное приложение позволяет администратору быть в курсе событий даже если он не находится непосредственно на рабочем месте. 
%%%%%%%%%%%%%%%%%%%%%%%%%%%%%%%%%%%%%%%%%%%%%%%%%%%%%%%%%%%%%%%%%%%%%%%
\section{Системы мониторинга общего назначения}
Представителями второй группы являются Elastic, Zabbix, Anomaly и др. Данные системы разрабатываются отдельно от систем хранения данных и предназначены для решения более общих задач, таких как сбор данных с большого количества источников, обработка данных, визуализация и поиск аномалий. 

\subsection{Elastic}

Система Elastic представляет из себя свободно распространяемый стек программ, предназначенный для сбора, хранения, обработки и отображения данных из разных источников. 
Состав программных продуктов Elastic представлен ниже: 
\begin{itemize*}
	\item{Elasticsearch – поисковая система с открытым исходным кодом, использует REST интерфейс и оперирует данными в формате JSON}
	\item{Kibana – инструмент позволяющий визуализировать данные в виде графиков, статистики и пр.}
	\item{Logstash - конвейер обработки данных на стороне сервера, который получает данные из нескольких источников одновременно, преобразует их, а затем отправляет в хранилище, например Elasticsearch}
	\item{Beats - платформа для легковесной отправки данных. Существует возможность отправлять данные с тысяч машин и систем в Logstash, используется для сбора данных}
	\item{ECE - Elastic Cloud Enterprise – предоставляет доступ для управления и контроля Elasticsearch и Kibana в любой инфраструктуре, управляя всем с одной консоли}
	\item{Machine learning – часть программного пакета X-Pack. Основной возможностью продукта является обнаружение аномалий временных рядов с использованием машинного обучения.}
\end{itemize*}

Наиболее интересным с точки зрения мониторинга параметров является последний пакет. Наилучшим применением данной технологии является обнаружение отклонений в значениях некоторых величин от нормального поведения. Чаще всего для этих целей используют правила, пороговые значения и различные простые статические методы. Однако такие способы зачастую неэффективны, так как они основываются на неверных статистических допущениях (Гауссово распределение и др.), а также не учитывают тренды (долгосрочное или периодическое изменение сигнала). Еще одним преимуществом пакета является возможность совместного анализа большого количества метрик, выявляя неявные зависимости. Предоставляется возможность вывода информации о всех аномалиях на один общий график. Анализ данных производится в реальном времени, возможно оповещение об обнаруженных аномалиях. Пакет полностью бесплатен начиная с 7 сентября 2018 года. 


\begin{figure}[htbp]
	\centering
	\includegraphics[width=\textwidth]{Elastic}
	\caption{Обнаружение аномалий в Elastic}
	\label{fig:Elastic}
\end{figure}

\subsection{Zabbix}
Zabbix - это программное обеспечение для мониторинга параметров сети, жизнеспособности и целостности серверов. Zabbix состоит из нескольких компонентов:
\begin{itemize*}
	\item{сервер мониторинга, который выполняет периодическое получение данных, обработку, анализ и запуск скриптов оповещения}
	\item{агенты - демоны, которые запускаются на отслеживаемых объектах и предоставляет данные серверу}
	\item{база данных (MySQL, PostgreSQL, SQLite или Oracle) для хранения накопленной информации}
	\item{веб-интерфейс на PHP, предоставляющий информацию о производительности и состоянии системы}
\end{itemize*}

\begin{figure}[!htbp]
	\centering
	\includegraphics[width=\textwidth]{Zabbix}
	\caption{Отображение парметров в графическом виде в системе Zabbix}
	\label{fig:Zabbix}
\end{figure}

Также Zabbix может работать без агента, для этого используются общие протоколы мониторинга, такие как протокол управления сетью.  Кроме того, мониторинг можно производить с помощью запуска внешних скриптов, встроенных проверок (http запрос, ping и др). 

\subsection{Anomaly} 
Сервис по диагностированию аномалий в реальном времени~\cite{Anomaly}. Данные для мониторинга загружаются в облако, где после обработки и процесса обучения становится доступным детектирование отклонений в поведении. При этом процесс обучения не останавливается и длится постоянно. После первичного обучения алгоритмы формируют простые паттерны определения ошибок, такие как максимумы и минимумы. Со временем происходит детализирование и выявление корреляций, трендов и др. Для обучения используется машинное обучение, нейронные сети, собственные математические модели. 
\begin{figure}[!h]
	\centering
	\includegraphics[width=\textwidth]{Anomaly}
	\caption{Пользовательский интерфейс Anomaly}
	\label{fig:Anomaly}
\end{figure}
При обнаружении аномалии пользователю отправляется сообщение (смс, почта) или выполняется собственно-назначенное действие.
\section{Постановка задачи}
Таким образом, существует большое количество различных систем мониторинга параметров, однако они обладают рядом недостатков:
\begin{itemize*}
	\item{Системы, поставляющиеся с оборудованием от производителя, работают только с их оборудованием, и, зачастую, не отображают данные в удобном виде}
	\item{Системы общего назначения максимально унифицированы под сбор и визуализацию всевозможных данных, поэтому такие системы не отображают иерархически систему, что усложняет определение последствий возникновения аномалий.  Кроме того, зачастую такие системы расположены в облаке, что делает невозможным их применение в закрытых(изолированных) решениях}
\end{itemize*}


%%\begin{table}
% Для таблиц с multirow и multicol необходимо вручную указать сдвиг для caption
%%\captionsetup{skip=5pt}
%%\caption{Example}
%%\centering
%%\begin{tabular}{|r|c|c|c|c|c|c|}
%%\hline
%%            \multirow{2}{*}{}
%%           & \multicolumn{2}{c|}{LLVM IR} 
%%           & \multicolumn{2}{c|}{PS} 
%%           & \multicolumn{2}{c|}{AI} \\ \cline{2-7}
%%           & SAT    & UNSAT   & SAT    & UNSAT   & SAT    & UNSAT   \\ \hline
%%beanstalkd & 356    & 252     & 360    & 161     & 360    & 247     \\ \hline
%%clib       & 599    & 258     & 960    & 234     & 960    & 449     \\ \hline
%%\end{tabular}
%%\label{table:checkResults}
%%\end{table}

%%%%%%%%%%%%%%%%%%%%%%%%%%%%%%%%%%%%%%%%%%%%%%%%%%%%%%%%%%%%%%%%%%%%%%%%%%%%%%%%
%%\section{bar}
%%%%%%%%%%%%%%%%%%%%%%%%%%%%%%%%%%%%%%%%%%%%%%%%%%%%%%%%%%%%%%%%%%%%%%%%%%%%%%%%

%%\blindtext
%%It is of great importance that you use correct references in your dissertation.
%%Resent studies show that it can increase the chances of successful defense
%%by as much as 3,17 percent~\cite{russian, java-book}.

%%\begin{table}[H]
%%	\caption{Название таблицы}
%%	\begin{center}
%%		\begin{tabular}{|l|l|}
%%			\hline
%%			top left & top right\\ \hline
%%			bot left & bot right\\ \hline
%%		\end{tabular}
%%		\label{tabular:tab_examp}
%%	\end{center}
%%\end{table}


